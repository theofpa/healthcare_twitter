\documentclass{article}

\begin{document}

\title{Healthcare twitter analysis.\\
Project breakdown.}
\author{Theofilos Papapanagiotou}

\maketitle

\section{Algorithms Research}
Classification of the tweets based on stages of a disease, like diagnosis,
treatment, etc.
Filter out the spam, most probably generated by drug advertisments, etc.
Train the clustering model using the Jan-May dataset, and test using
the June dataset.
Produce a vocabulary which might help the disease stage classification,
like the sentiment analysis of Introduction to Data Science [1].

\section{Technology / Framework Research}
Use the experience gained in Octave from Andrew Ng's Machine Learning [2] and
using R from Data Science Specialization [3], to learn another language
libraries on Data Science, Python.

The given dataset contains date, user, url and tweet in a csv, which are
sufficient for the analysis. It would be cool though, to load the full tweet
metadata in a mongo calling the twitter API by tweetId. Mongolab free 0.5GB db
might be sufficient for the training dataset.

If we need to scale more and mapreduce our algorithm, utilization of a few
hadoop instances in ec2 might be cooler.

\section{Business/ Domain Research}
Search heathdata.gov, datahub.io, enigma.io, gapminder.org for open data on
healthcare to correlate the disease related tweets as proposed by Pratik.

\section{Visualization Research}
Extend the gained ggplot2 knowledge in Python world, to produce high quality
graphs and present the results of the project.
Make sure that everything is reproducible and delivered in iPython Notebook.

\end{document}
